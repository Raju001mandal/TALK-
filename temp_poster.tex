% Unofficial University of Cambridge Poster Template
% https://github.com/andiac/gemini-cam
% a fork of https://github.com/anishathalye/gemini
% also refer to https://github.com/k4rtik/uchicago-poster
% poster to be presented in "Future Perspectives on QFT and Strings" in IISER Pune.
\documentclass[final]{beamer}
\title{All S Invariant Gluon OPEs on the Celestial Sphere}
\author{Raju Mandal}
\date{\today}
\institute{NISER, Bhubaneswar}
%================themes and packages========================
\usepackage[T1]{fontenc}
\usepackage{lmodern}
\usepackage[orientation=portrait,size=a0,scale=1.0]{beamerposter}
\usetheme{gemini}
\usecolortheme{nott}
\usepackage{graphicx}
\usepackage{booktabs}
\usepackage{tikz}
\usepackage{pgfplots}
\pgfplotsset{compat=1.14}
\usepackage{anyfontsize}
%=======added from talk slide==================================

\usepackage{tensor}
\usepackage{pifont}  %for more bullet shapes
\usepackage{mathrsfs}
\usepackage{dirtytalk}

%====================================================================
% ====================
% Lengths
% ====================
% If you have N columns, choose \sepwidth and \colwidth such that
% (N+1)*\sepwidth + N*\colwidth = \paperwidth
\newlength{\sepwidth}
\newlength{\colwidth}
\setlength{\sepwidth}{0.025\paperwidth}
\setlength{\colwidth}{0.45\paperwidth}

\newcommand{\separatorcolumn}{\begin{column}{\sepwidth}\end{column}}
\newcommand{\chaptype}[1]{{#1}}
%========================================================================
% ====================
% Title
% ====================

\title{An Infinite Family of S Invariant Theories on the Celestial Sphere}

\author{Shamik Banerjee \inst{1,2} \and \textbf{Raju Mandal} \inst{1,2} \and Sagnik Misra \inst{1,2} \and Sudhakar Panda \inst{3} \and Partha Paul \inst{4}}

\institute[shortinst]{\hspace{-20cm} 2024 \hspace{17cm} \inst{1} HBNI, Mumbai \samelineand \inst{2} \textbf{NISER, Bhubaneswar} \samelineand \inst{3} CCSP, Gurugram \samelineand \inst{4} IISC, Bangalore}

% ====================
% Footer (optional)
% ====================

%\footercontent{
	%  \href{https://www.example.com}{https://www.example.com} \hfill
	% ABC Conference 2025, New York --- XYZ-1234 \hfill
	% \href{mailto:agajan.torayev@example.com}
	%{agajan.torayev@example.com}}
% (can be left out to remove footer)


% ====================
% Logo (optional)
% ====================

% use this to include logos on the left and/or right side of the header:
\logoright{\includegraphics[height=7cm]{/home/jura/Dropbox/LATEX/iiserpune/niser2.png}}
\logoleft{\includegraphics[height=6.5cm]{/home/jura/Dropbox/LATEX/iiserpune/iiserpunelogo.jpg}}
%\logo{\includegraphics[height=10cm]{/home/jura/Dropbox/LATEX/iiserpune/iiserpunelogo.jpg}}
% ====================
% Body
% ====================
\newcommand{\Poincare}{Poincar\'e\xspace}

\begin{document}

\begin{frame}[t]
	\begin{columns}[t]
		\separatorcolumn
		
		\begin{column}{\colwidth}
			
	    %================================================================
	    \begin{block}{Celestial Holography, a Putative Dual of QG in AFS}
	    	Quantum theories of Gravity in (3+1)-D AFS  $\Longleftrightarrow$ 2D CCFTs on the Celestial sphere at null infinity.
	    	\begin{figure}
	    		\includegraphics[scale=.9]{celestial_img1.png}
	    	\end{figure}
	    	\hspace{26cm}\small{[Courtesy: Laura Donnay]}
	    \end{block}
	    %================================================================
		\begin{block}{Asymptotic Symmetry Algebras for Gravity and Gauge Theories in AFS}
			 \begin{itemize}
			 	\item[\ding{71}] \boxed{\textbf{Conformally soft theorems}} $\longrightarrow$ Infinite dimensional non-trivial symmetries in Gauge theories and Gravity in 4D AFS.
			 	\\
			 	\vspace{4mm}
			 	\item[\ding{71}]  \boxed{\textbf{Gravity}} $\longrightarrow$  \boxed{\textbf{wedge subalgebra of $w_{1+\infty}$ algebra}}.
			 	\hspace{4cm} \textcolor{red}{\small(Guevara,Himwich,Pate and Strominger '21)} 
			 	\vspace{4mm}
			 	\item[\ding{71}] \boxed{\textbf{Gauge Theories}} $\longrightarrow$ \boxed{ \textbf{ $S$ algebra } } 
			 \end{itemize}
		\end{block}	
	    
	    %=============================================================
	    \begin{block}{Motivation and the Goal}
	    	\begin{itemize}
	    		\item [\ding{71}] \boxed{\textbf{\textcolor{red}{Banerjee, Kulkarni and Paul '23}}}
	    		$\Rightarrow$ computed $G^{+}G^{+}$ OPE and KZ-type null states of any $w_{1+\infty}$ invariant theories \hspace{16cm}\textcolor{blue}{2301.13225}; \textcolor{red}{2311.06485}
	    		\vspace{2mm}\\
	    		$\Rightarrow$ existence of a discrete infinite family of $w_{1+\infty}$ invariant theories on the celestial sphere.
	    		\vspace{2mm}\\
	    		\item[\ding{71}] \boxed{\textbf{\textcolor{red}{Goal}}}
	    		$\Rightarrow$
	    		Our goal is to classify the theories which are invariant under \textbf{S algebra} and also to find the \textbf{KZ}-type null states of these theories.  \hspace{5.6cm} \textcolor{blue}{2311.16796}
	    		\\
	    		\hspace{20cm} \textcolor{red}{\small{(Banerjee, RM, Misra, Panda and Paul '23)}}
	    	\end{itemize}
	    \end{block}
		%=================================================================	
		\begin{block}{Gluons and the S Algebra}
				The \textcolor{red}{S algebra} is obtained from the \textcolor{red}{singular part of the OPE} i.e. 
			\\
			\hspace{21cm}\textcolor{red}{\small{Guevara, Himwich, Pate and Strominger '21 }}
			\begin{equation}
				\label{eq:soft4}
				\begin{split}
					&\mathcal{O}^{a,+}_{\Delta_{1}}(z_{1},\bar{z}_{1}) \mathcal{O}^{b,+}_{\Delta_{2}}(z_{2},\bar{z}_{2})
					\sim -\frac{i\tensor{f}{^a^b_c}}{z_{12}}\sum_{n=0}^{\infty}B(\Delta_{1}+n-1,\Delta_{2}-1)\frac{\bar{z}^{n}_{12}}{n!}\bar\partial^{n}_{2}\mathcal{O}^{c,+}_{\Delta_{1}+\Delta_{2}-1}(z_{2},\bar{z}_{2}).
				\end{split}
			\end{equation}
			\\
			\textcolor{red}{Soft gluons :}
			\begin{equation}
				\label{eq:soft1}
				R^{k,a}(z,\bar{z}) = \lim_{\Delta \to k}(\Delta-k)O^{a,+}_{\Delta}(z,\bar{z}) , \hspace{1cm} k=1,0,-1,...
			\end{equation}
			\textcolor{red}{Holomorphic soft gluon currents :}
			\begin{equation}
				\label{eq:soft2}
				R^{k,a}(z,\bar{z}) =\sum_{n=\frac{k-1}{2}}^{\frac{1-k}{2}}\frac{R^{k,a}_{n}(z)}{\bar{z}^{n+\frac{k-1}{2}}}
			\end{equation}
			Modes of the Holomorphic currents :
		\begin{equation}
			\label{eq:soft3}
			R^{k,a}_{n}(z)=\sum_{\alpha\in \mathbb{Z}-\frac{k+1}{2}} \frac{R^{k,a}_{\alpha,n}}{z^{\alpha+\frac{k+1}{2}}}
		\end{equation}
		Algebra :
		\begin{equation}
			\small
			\label{eq:soft5}
			[R^{k,a}_{\alpha,m},R^{l,b}_{\beta,n}]=-i\tensor{f}{^a^b_c}\frac{(\frac{1-k}{2}-m+\frac{1-l}{2}-n)!}{(\frac{1-k}{2}-m)!(\frac{1-l}{2}-n)!}	
			\frac{(\frac{1-k}{2}+m+\frac{1-l}{2}+n)!}{(\frac{1-k}{2}+m)!(\frac{1-l}{2}+n)!}R^{k+l-1,c}_{\alpha+\beta,m+n}
		\end{equation}
		Redefinition :
		\begin{equation}
			\label{eq:soft6}
			S^{q,a}_{\alpha,m}=(q-m-1)!(q+m-1)!R^{3-2q,a}_{\alpha,m}
		\end{equation}
		\textcolor{blue}{\textbf{S Algebra :}}
		\begin{equation}
			\label{eq:soft7}
			[S^{p,a}_{\alpha,m},S^{q,b}_{\beta,n}]=-if^{abc}S^{p+q-1,c}_{\alpha+\beta,m+n}
		\end{equation}
		\end{block}
		%==================================================================
		\begin{block}{General Structure of Gluon-Gluon OPE}
			\begin{equation}
				\begin{split}
					\label{eq:ope1}
					&\mathcal{O}^{a,+}_{\Delta_{1}}(z_{1},\bar{z}_{1}) \mathcal{O}^{b,+}_{\Delta_{2}}(z_{2},\bar{z}_{2})\\
					&\hspace{-2mm}
					=-\frac{i\tensor{f}{^a^b_c}}{z_{12}}\sum_{n=0}^{\infty}B(\Delta_{1}+n-1,\Delta_{2}-1)\frac{\bar{z}^{n}_{12}}{n!}\bar\partial^{n}_{2}\mathcal{O}^{c,+}_{\Delta_{1}+\Delta_{2}-1}(z_{2},\bar{z}_{2})
					+\sum_{p,q=0}^{\infty}\sum_{k=1}^{\tilde{n}_{p,q}}z^{p}_{12}\bar{z}^{q}_{12}\textcolor{red}{C^{k}_{p,q}}(\Delta_{1},\Delta_{2})\textcolor{red}{\tilde{\mathcal{O}}^{ab}_{k,p,q}}(z_{2},\bar{z}_{2}).
				\end{split}
			\end{equation}
			\textcolor{blue}{Task is to determine}: 
			\begin{itemize}
				\item [\ding{45}] The OPE coefficients $C^{k}_{p,q}$ and
				\vspace{2mm}
				\item[\ding{45}] the S-algebra descendants $\tilde{\mathcal{O}}^{ab}_{k,p,q}$ of a positive helicity soft gluon.
			\end{itemize}
		\end{block}
		%==================================================================
		\begin{block}{Strategy}
			\begin{itemize}
				\item[\ding{45}] We consider S invariant theories for all of which S-algebra is universal $\Rightarrow$ Existence of a Master OPE .
				\vspace{3mm}
				\item[\ding{45}] We know that tree-level MHV sector of the pure YM theory is an example of such S invariant theories.
				\vspace{3mm}
				\item [\ding{45}] This Master OPE inserted in a MHV gluon scattering amplitude $\Rightarrow$ known MHV OPE.
				\vspace{3mm}
				\item [\ding{45}] \boxed{\textrm{\textbf{Master OPE = MHV-sector OPE + R}}}
				\vspace{3mm}
				\item [\ding{45}] $\textrm{R}$ should vanish inside MHV scattering amplitude $\Rightarrow$ $\textrm{R}$ is a lin. combination of MHV null states.
				\vspace{3mm} 
				\item [\ding{45}] $\textrm{R}$ consists only non-singular terms.
			\end{itemize}
		\end{block}
		%==================================================================
	\end{column}
		\separatorcolumn
		
		\begin{column}{\colwidth}
		%===================================================================
		\begin{block}{OPEs in Terms of MHV Null States}
				Using the above arguments we can rewrite $\eqref{eq:ope1}$ as,
			\begin{equation}
				\label{eq:ope2}
				\hspace{-2mm}\boxed{
					\begin{split}
						&\mathcal{O}^{a,+}_{\Delta_{1}}(z_{1},\bar{z}_{1}) \mathcal{O}^{b,+}_{\Delta_{2}}(z_{2},\bar{z}_{2})|_{\textrm{Any Theory}}=\mathcal{O}^{a,+}_{\Delta_{1}}(z_{1},\bar{z}_{1}) \mathcal{O}^{b,+}_{\Delta_{2}}(z_{2},\bar{z}_{2})|_{\textrm{MHV}}\\
						& \hspace{15cm} +\sum_{p,q=0}^{\infty}z^{p}_{12}\bar{z}^{q}_{12}\sum_{i=1}^{\tilde{n}_{p,q}}\tilde{C}^{k}_{p,q}(\Delta_{1},\Delta_{2})M^{a,b}_{k,p,q}(\Delta_1,\Delta_2,z_{2},\bar{z}_{2}).
				\end{split}}
			\end{equation}
		\end{block}
		%==================================================================
		\begin{block}{MHV Null States at $\mathcal{O}(1)$}
				The general null state at $\mathcal{O}(1)$ in the MHV-sector is given by
			\begin{equation}
				\label{eq:mhv1}
				\begin{split}
					&\Psi^{ab}_{j}(\Delta)=R^{-j,a}_{\frac{j-1}{2},\frac{j+1}{2}}\mathcal{O}^{b}_{\Delta+j,+}-\frac{(-1)^{j}j}{\Gamma(j+2)}\frac{\Gamma(\Delta+j-1)}{\Gamma(\Delta-2)}R^{1,a}_{-1,0}\mathcal{O}^{b}_{\Delta-1,+}
					\\
					&\hspace{5cm}
					-\frac{(-1)^{j}}{\Gamma(j+1)}\frac{\Gamma(\Delta+j-1)}{\Gamma(\Delta-1)}R^{0,a}_{-1/2,1/2}\mathcal{O}^{b}_{\Delta,+}
				\end{split}
			\end{equation}
			where $j=1,2,3,...$
			\vspace{2mm}\\
			\textcolor{red}{Let's consider the following basis :}
			\begin{equation}
				\label{eq:mhv2}
				\begin{split}
					M^{ab}_{k}(\Delta)= \sum_{i=1}^{k} \frac{1}{\Gamma(k-i+1)} \frac{\Gamma(\Delta+k-1)}{\Gamma(\Delta+i-1)}\Psi^{ab}_{i}(\Delta).
				\end{split}
			\end{equation}
		
		\end{block}
		%==================================================================
		
		%==================================================================
		\begin{block}{Action of the S algebra on the MHV Null States}
			\textcolor{blue}{\textbf{Action of the Leading Soft Gluon modes:}}
			\begin{equation}
				\label{eq:act1}
				R^{1,a}_{0,0}M^{bc}_{k}(\Delta)=-if^{abd}M^{dc}_{k}(\Delta)-if^{acd}M^{bd}_{k}
			\end{equation}
			\begin{equation}
				\label{eq:act2}
				R^{1,a}_{n,0}M^{bc}_{k}(\Delta)=0 , n>0
			\end{equation}
			\textcolor{blue}{\textbf{Action of the Subleading Soft Gluon mode :}}
			\begin{equation}
				\label{eq:new7}
				\begin{split}
					&[R^{0,a}_{1/2,1/2},M^{bc}_{k}(\Delta)]= -if^{abd}(k+2)M^{dc}_{k+1}(\Delta-1) +(\Delta+k-2) \Bigg\{if^{acd}M^{bd}_{k}(\Delta-1)+if^{abd}M^{dc}_{k}(\Delta-1)\Bigg\}
				\end{split}.
			\end{equation}
		\end{block}
	%===================================================================
	\begin{block}{Higher Spin Currents in S Algebra}
		\centering
		\includegraphics[scale=1]{fig.pdf}
	\end{block}
	%======================================================================
	\begin{block}{Building up S invariant OPEs}
		\textcolor{blue}{\textbf{Observation :}}
		\begin{itemize}
			\item[\ding{224}]
			Consider the following set of null states,
			\begin{equation}
				\label{eq:s1}
				M^{bc}_{k}(\Delta) , k=1,2,3,...,n.
			\end{equation}
			\item[\ding{224}]
			Action of $R^{0,a}_{1/2,1/2}$ on the null states is closed if we set
			\begin{equation}
				\label{eq:s2}
				M^{ab}_{k+1}(\Delta)=0, k \ge n \ge 0.
			\end{equation}
		\end{itemize}
		\textcolor{blue}{\textbf{Inference :}} 
		\begin{itemize}
			\item[\ding{224}]
			We can get an S invariant OPE if we consider the finite set of null states $\eqref{eq:s1}$.
		\end{itemize}
	    \begin{equation}
	    	\label{eq:s3}
	    	\hspace{-4mm}\boxed{
	    		\begin{split}
	    			&\mathcal{O}^{a}_{\Delta_{1},+}(z,\bar{z})\mathcal{O}^{b}_{\Delta_{2},+}(0,0)|_{\mathcal{O}(1)}
	    			=\mathcal{O}^{a}_{\Delta_{1},+}(z,\bar{z})\mathcal{O}^{b}_{\Delta_{2},+}(0,0)\Bigg|^{MHV}_{\mathcal{O}(1)}+\sum_{k=1}^{n}B(\Delta_{1}+k,\Delta_{2}-1)M^{ab}_{k}(\Delta_{1}+\Delta_{2})
	    	\end{split}}
	    \end{equation}
    \begin{itemize}
    	\item[\ding{224}] S invariance does not fix the value of integer n.
    	\item [\ding{224}] Hence, different choices of the integer n give rise to a discrete infinite family of S-invariant OPEs. 
    \end{itemize}
	\end{block}
%==========================================================================
 \begin{block}{Knizhnik-Zamolodchikov Type Null States}
\begin{equation}
\label{eq:kz2}
\boxed{
K^{a}(\Delta)=\xi^{a}(\Delta)-i\sum_{k=1}^{n}M^{a}_{k}(\Delta+1)},
\end{equation}
where
\begin{equation}
\label{eq:kz3}
\xi^{a}(\Delta)=C_{A}L_{-1}\mathcal{O}^{a,+}_{\Delta}-(\Delta+1)R^{1,b}_{-1,0}R^{1,b}_{0,0}\mathcal{O}^{a,+}_{\Delta}-R^{0,b}_{-\frac{1}{2},\frac{1}{2}}R^{1,b}_{0,0}\mathcal{O}^{a,+}_{\Delta+1}
\end{equation}
and
\begin{equation}
	M^{a}_{k}(\Delta)=f^{abc}M^{bc}_{k}.
\end{equation}
 \end{block}
%=================================================================
\begin{block}{References}
	 S. Banerjee, R. Mandal, S. Misra, S. Panda and P. Paul, “All S invariant gluon OPEs on the
	celestial sphere,” [arXiv:2311.16796 [hep-th]]
	\footnotesize{\bibliographystyle{plain}\bibliography{poster}}
	\vspace{1mm}
	\textcolor{red}{Email} : raju.mandal@niser.ac.in
\end{block}
%==================================================================
		\end{column}
	\separatorcolumn
	
\end{columns} 
\end{frame}


%=====================================================================

	
\end{document}
